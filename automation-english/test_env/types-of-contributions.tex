\hypertarget{ux8d21ux732eux7c7bux578b}{%
\subsubsection{贡献类型}\label{ux8d21ux732eux7c7bux578b}}

问题:正在进行哪些类型的贡献?

\hypertarget{ux63cfux8ff0}{%
\paragraph{描述}\label{ux63cfux8ff0}}

多元化的贡献能够使开源项目健康发展。在很多项目中,有些社区成员并不编写代码,但他们同样做出了有价值的贡献,比如管理社区、区分错误、宣传项目、支持用户或以其他方式提供帮助。

\hypertarget{ux76eeux6807}{%
\paragraph{目标}\label{ux76eeux6807}}

多样的贡献类型表明项目是成熟全面的,包含足够的活动来支持项目的所有方面,并且提供多种贡献类型的晋升渠道,让拥有编码之外的不同专长的人员也能够发展到领导层。

\hypertarget{ux5b9eux73b0}{%
\paragraph{实现}\label{ux5b9eux73b0}}

如何对贡献进行定义、量化、跟踪和公布是一个具有挑战性的问题。
每个项目的答案可能都是独一无二的,以下建议仅作为抛砖引玉。作为一个通用指南,很难将不同的贡献类型相互比较,应单独考量。

\begin{itemize}
\tightlist
\item
  以下列表可以帮助确定贡献类型:

  \begin{itemize}
  \tightlist
  \item
    编写代码
  \item
    审查代码
  \item
    错误分类
  \item
    质量保证和测试
  \item
    安全相关活动
  \item
    本地化和翻译
  \item
    事件组织
  \item
    文档编写
  \item
    社区建设和管理
  \item
    教学和教程构建
  \item
    故障排除和支持
  \item
    创意作品和设计
  \item
    用户界面、用户体验和易用性
  \item
    社交媒体管理
  \item
    用户支持和问题解答
  \item
    撰写文章
  \item
    公共关系 - 技术媒体采访
  \item
    事件发言
  \item
    营销与活动宣传
  \item
    网站开发
  \item
    法律委员会
  \item
    财务管理
  \end{itemize}
\end{itemize}

\hypertarget{ux6570ux636eux6536ux96c6ux7b56ux7565}{%
\subparagraph{数据收集策略}\label{ux6570ux636eux6536ux96c6ux7b56ux7565}}

\begin{itemize}
\item
  **采访或调查:**让社区成员认可他人的贡献,找出过去没有考虑到的贡献类型。

  \begin{itemize}
  \tightlist
  \item
    您想表彰谁在项目中的贡献? 他们贡献了什么?
  \end{itemize}
\item
  **观察项目:**找出和认可项目不同部分的负责人。

  \begin{itemize}
  \tightlist
  \item
    在项目网站或代码仓库中列出了哪些负责人?
  \end{itemize}
\item
  **捕获非代码贡献:**通过问题跟踪器等专用系统跟踪贡献。

  \begin{itemize}
  \tightlist
  \item
    日志记录可以包含项目要了解的关于非代码贡献的定制化信息,用以识别工作量。
  \item
    通过沟通渠道活动的代理贡献。 例如,如果质量保证成员 (QA)
    拥有自己的邮件列表,就可以通过邮件列表活动来代理衡量围绕 QA
    贡献的活动。
  \end{itemize}
\item
  **收集跟踪数据:**通过协作工具日志数据衡量贡献。

  \begin{itemize}
  \tightlist
  \item
    例如,可以从源代码仓库计算代码贡献,可以从维基编辑历史记录计算维基贡献,可以从电子邮件归档计算电子邮件消息
  \end{itemize}
\item
  **自动分类:**训练人工智能 (AI) 机器人来检测贡献并对其分类。

  \begin{itemize}
  \tightlist
  \item
    AI
    机器人可以协助对贡献进行分类,例如,帮助请求与提供的支持,或功能请求与错误报告,尤其是以上均在同一个问题跟踪器中完成的情况。
  \end{itemize}
\end{itemize}

\emph{其他考量:}

\begin{itemize}
\tightlist
\item
  特别是对于自动报告,允许社区成员选择退出并不出现在贡献报告上。
\item
  承认对贡献类型的捕捉不完善,并明确说明收集了哪些类型的贡献。
\item
  随着项目发展,贡献类型的收集方法需要作出调整。
  例如,交换国际化库时,围绕本地化的项目活动可能会在变化前后产生不同的指标。
\item
  大规模挖掘贡献类型时,要考虑机器人的活动。
\end{itemize}

\hypertarget{ux53c2ux8003ux8d44ux6599}{%
\paragraph{参考资料}\label{ux53c2ux8003ux8d44ux6599}}

\begin{itemize}
\tightlist
\item
  \href{https://medium.com/@sunnydeveloper/revisiting-the-word-recognition-in-foss-and-the-dream-of-open-credentials-d15385d49447}{https://medium.com/@sunnydeveloper/revisiting-the-word-recognition-in-foss-and-the-dream-of-open-credentials-d15385d49447}
\item
  \href{https://24pullrequests.com/contributing}{https://24pullrequests.com/contributing}
\item
  \href{https://smartbear.com/blog/test-and-monitor/14-ways-to-contribute-to-open-source-without-being/}{https://smartbear.com/blog/test-and-monitor/14-ways-to-contribute-to-open-source-without-being/}
\item
  \href{https://wiki.openstack.org/wiki/AUCRecognition}{https://wiki.openstack.org/wiki/AUCRecognition}
\item
  \href{https://www.drupal.org/drupalorg/blog/a-guide-to-issue-credits-and-the-drupal.org-marketplace}{https://www.drupal.org/drupalorg/blog/a-guide-to-issue-credits-and-the-drupal.org-marketplace}
\end{itemize}
