\hypertarget{job-opportunities}{%
\section{Job Opportunities}\label{job-opportunities}}

Question: How many job postings request skills with technologies from a
project?

\hypertarget{description}{%
\subsection{Description}\label{description}}

A common way for open source contributors to earn a living wage is to be
employed by a company or be a self-employed or freelance developer.
Skills in a specific project may improve a job applicant's prospects of
getting a job. The most obvious indicator for demand related to a skill
learned in a specific open source project is when that project or its
technology is included in job postings.

\hypertarget{objectives}{%
\subsection{Objectives}\label{objectives}}

The metric gives contributors a sense of how much skills learned in a
specific open source project are valued by companies.

\hypertarget{implementation}{%
\subsection{Implementation}\label{implementation}}

To obtain this metric on a job search platform (e.g., LinkedIn, Indeed,
or Dice), go to the job search and type in the name of the open source
project. The number of returned job postings is the metric. Periodically
collecting the metric through an API of a job search platform and
storing the results allows to see trends.

\hypertarget{filters}{%
\subsubsection{Filters}\label{filters}}

\begin{itemize}
\tightlist
\item
  Age of job posting; postings get stale and may not be removed when
  filled
\end{itemize}

\hypertarget{visualizations}{%
\subsubsection{Visualizations}\label{visualizations}}

The metric can be extended by looking at:

\begin{itemize}
\tightlist
\item
  Salary ranges for jobs returned
\item
  Level of seniority for jobs returned
\item
  Availability of jobs like on-site or off-site
\item
  Location of job
\item
  Geography
\end{itemize}

\hypertarget{references}{%
\subsection{References}\label{references}}

\begin{itemize}
\tightlist
\item
  LinkedIn Job Search API:
  \url{https://developer.linkedin.com/docs/v1/jobs/job-search-api\#}
\item
  Indeed Job Search API:
  \url{https://opensource.indeedeng.io/api-documentation/docs/job-search/}
\item
  Dice.com Job Search API:
  \url{http://www.dice.com/external/content/documentation/api.html}
\item
  Monster Job Search API: \url{https://partner.monster.com/job-search}
\item
  Ziprecruiter API (Requires Partnership):
  \url{https://www.ziprecruiter.com/zipsearch}
\end{itemize}

\emph{Note:} This metric is limited to individual projects but
engagement in open source can be beneficial for other reasons. This
metric could be tweaked to look beyond a single project and instead use
related skills such as programming languages, processes, open source
experience, or frameworks as search parameters for jobs.
